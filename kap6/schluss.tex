\chapter{Zusammenfassung}

Seit dem Aufkommen der Web Graphics Library im Jahr 2009 ist eine neue Dynamik im Web3D-Umfeld entstanden. Mit der heutigen Verfügbarkeit von WebGL in nahezu jedem Webbrowser ist die Darstellung hardwarebeschleunigter 3D-Grafik heute ohne Plugins möglich. Mit X3DOM hat sich innerhalb der letzten Jahre ein umfangreiches JavaScript-Framework entwickelt, welches die Nutzung des freien X3D-Standards zur Deklaration von 3D-Szenen in HTML ermöglicht. Behr et al. erhoffen sich, langfristig eine native Unterstützung von X3D innerhalb der Browser ähnlich zu SVG zu erzielen. Auch auf Seite von WebGL sind seit dessen Erscheinung eine Vielzahl von Frameworks entstanden, welche die Programmierung anspruchsvoller 3D-Anwendungen mit großer Flexibilität erlauben. Durch Abstraktion der Funktionaliät der Hardware-nahen Grafikbibliothek wird deren Benutzung stark vereinfacht.

Wie die Evaluation gezeigt hat, ähneln sich X3DOM und WebGL in vielen Aspekten wie der Browser- und Plattformunterstützung und den Möglichkeiten hinsichtlich der Benutzerinteraktion sehr. Auch der Import bestehender 3D-Daten ist bei beiden Ansätzen einfach zu realisieren. Der hauptsächliche Unterschied der zwei betrachteten Technologien liegt in ihrer paradigmatischen Grundlage. Zwar ermöglicht es der deklarative Stil von X3DOM sehr leicht, 3D-Szenen auch ohne Programmierkenntnisse umzusetzen, jedoch schränkt dies die Flexiblität des Entwicklers gleichzeitig ein, da nicht jeder Aspekt der Darstellung bis ins letzte Detail beeinflusst werden kann. Je nach konkreter Anwendung und Vorwissen des Software-Entwicklers muss die Wahl der 3D-Technologie somit von Fall zu Fall entschieden werden.

Die von Ortiz 2010 hauptsächlich gesehenen Probleme, die Web3D davon abhalten, sich im WWW stärker zu etablieren, -- der Mangel an Standardisierung und die Notwendigkeit von Browser-Plugins -- können im Jahr 2014 als überwunden betrachtet werden. Es bleibt abzuwarten, inwieweit sich WebGL in den nächsten Jahren weiterentwickeln wird und ob es X3D ebenso schaffen wird, eine native Unterstützung innerhalb der Webbrowser zu erreichen.