Vor der näheren Betrachtung von X3D und WebGL soll zunächst deren Ursprung dargelegt werden, um die grundlegenden Konzepte dieser heutigen Technologien zu erläutern. Die Geschichte von Web3D reicht über zwei Jahrzehnte zurück, in denen es eine Vielzahl von Bestrebungen verschiedenster Institutionen gab, 3D-""Grafik im World Wide Web (WWW) zu etablieren. Die überwiegende Mehrheit dieser Ansätze scheiterte jedoch und konnte nie eine erwähnenswerte Bedeutung im WWW erlangen \autocite{Evans201443}. Erst innerhalb der letzten Jahre entwickelte sich durch das Aufkommen WebGLs eine neue Dynamik innerhalb dieses Felds.

Die ersten Anfänge der webbasierten 3D-""Computergrafik gehen bis ins Frühjahr 1994 zurück, als der Erfinder des noch jungen World Wide Webs, Tim Berners-Lee, öffentlich dazu aufrief eine Spezifikation für virtuelle Realität zu entwickeln \autocite{Parisi:2012}. Der Informatiker Dave Raggett, der seit 1992 an der Entwicklung zahlreicher Web-Standards richtungsweisend beteiligt war \autocite{PEOPLE_OF_THE_W3C_RAGGETT}, reichte daraufhin seine Publikation \enquote{\emph{Extending WWW to support Platform Independent Virtual Reality}} für die im selben Jahr stattfindende erste internationale World Wide Web Konferenz (\emph{WWW1}) in Genf ein. Ragget prägte innerhalb dieser Arbeit den Begriff \emph{Virtual Reality Markup Language} (VRML\footnote{Gesprochen \emph{vermal}.}) und beschrieb seine Vision von webbasierter virtueller Realität  \autocite{VMRL_RAGGETT}: Eine Auszeichnungssprache auf Basis der \emph{Standard Generalized Markup Language} (SGML) soll es ermöglichen, hierarchische 3D-""Szenen auf einer abstrakten Ebene deklarativ zu beschreiben.

Auf der WWW1 im Mai 1994 bildete sich daraufhin auf Initiave von Ragget und Berners-Lee hin die sogenannte VRML BOF\footnote{Eine sogenannte \emph{Birds of a feather} (BOF) stellt eine informelle Diskussionsrunde von Experten zu einem spezifischen Thema dar. Diese finden insbesondere im Umfeld der Internet Engineering Task Force häufig Anwendung \autocite{RFC5434} \autocite{IETF_BOF_PROCEDURES}.}. Ziel dieser Expertenrunde war die Diskussion bereits bestehender Ansätze von 3D-""Visualisierung mit Fokus auf der Interaktion mit dem WWW. Der auf diesem Gebiet Pionierarbeit leistende Ingenieur Mark Pesce stellte hierbei einen in Zusammenarbeit mit dem Softwareentwickler Tony Parisi entwickelten einfachen Prototypen vor \autocite{Parisi:2012}. Die Demonstration zeigte eine einfache, dreidimensionale Darstellung eines Würfels und ermöglichte die Interaktion mit dem WWW, indem eine URL aufgerufen wurde, wenn man auf die Geometrie klickte. Der Grundstein für die weitere Entwicklung der Virtual Reality Modeling Language im Besonderen und der webbasierten Computergrafik im Allgemeinen war somit gelegt. Um den Bezug zur Computergrafik zu verdeutlichen, wurde die Sprache kurz darauf in Virtual Reality \emph{Modeling} Language umbenannt \autocite{VRML_10_SPECIFICATION}.

Die daraufhin gegründete, schnell wachsende Arbeitsgruppe rund um die \emph{www-vrml} Mailing-Liste einigte sich auf das ehrgeizige Ziel, eine erste Spezifikation noch innerhalb des Jahres 1994 fertigzustellen \autocite{VRML_10_SPECIFICATION}. Aufgrund der guten Basis, die das \emph{Inventor Format} des amerikanischen Computerherstellers \emph{Silicon Graphics} bot, entschied man sich für dieses als Ausgangspunkt für die weitere Entwicklung der Sprache. Tatsächlich konnte VRML 1.0 im November vollendet werden.

Zwei Jahre später wurde deren deutlich ausgereiftere Version 2.0 der Spezifikation fertiggestellt und 1997 zu einem ISO-Standard\footnote{ISO/IEC 14772-1:1997.} \autocite{VRML_20_ISO} erhoben. Zur stetigen Weiterentwicklung und Schutz dieser neuen, als \emph{VRML97} bekannt gewordenen, offenen Spezifikation wurde kurze Zeit später das \emph{VRML Consortium} gegründet. Die Non-Profit-Organisation, die inzwischen den Namen \emph{Web3D Consortium} trägt, betreut bis heute die Entwicklung des VRML-Nachfolgers X3D und treibt dessen Verbreitung voran. Sie setzt sich aus Unternehmen, akademischen Institutionen und Einzelpersonen zusammen \autocite{WEB3D_CONSORTIUM_ABOUT}.

Trotz der hohen Erwartungen konnte sich die VRML nie in dem Maße im WWW durchsetzen, wie es sich deren Entwickler zunächst erwartet hatten. Die Gründe hierfür sind vielfältig: Zunächst war die Installation eines Browser-Plugins zwingend notwendig, um das Format innerhalb des Webbrowsers darstellen zu können. Die Hardware des durchschnittligen damaligen Personal Computers wies darüber hinaus in vielen Fällen eine unzureichende Leistung, sodass rechenaufwändige 3D-""Szenerien nicht zu realisieren waren. Weiterhin war die Geschwindigkeit damaliger Internetanschlüsse zu gering, um die relativ großen Datenmengen, die bei Computergrafik typischerweise anfallen, bewältigen zu können.
