\chapter{Das World Wide Web als Anwendungsplattform der Zukunft}
\label{CHAP:INTRODUCTION}

Durch die zunehmend größere Verbreitung moderner Webstandards wie HTML5 und damit assoziierter Technologien hat sich das World Wide Web (WWW) von einem ehemals simplen Dokumentenbetrachtungssystem zu einer Plattform für vielfältige \emph{Rich Internet Applications} (RIA) entwickelt \autocite{Taivalsaari:2001:Saga_continues}. \textcite{Anttonen:2011:TWR:1982185.1982357} unterteilen diese Entwicklung in drei grobe Phasen:

Während das Web zu Anfang aus statischen, textorientierten HTML-Dokumente bestand, leitet das Aufkommen JavaScripts 1995 die zweite Phase mit immer mehr interaktiven Elementen ein. Proprietäre, weit verbreitete Browser-Plugins wie \emph{Flash}, \emph{Shockwave} und \emph{Quicktime} ermöglichen zusätzlich zu unbewegten Bildern nun auch Animationen und die Einbettung von Video- und Audio-Inhalten.\newline
Die dritte Stufe stellt schließlich den derzeitigen Übergang zu hochinteraktiven Web-""Anwendungen dar, die klassischen Desktop-""Applikationen in ihrem Nutzererlebnis (\emph{User Experience}) immer mehr ähneln.
\textcite{Taivalsaari:2011:Death_of_Binary_Software} vertreten in ihrer Publikation \enquote{\emph{The Death of Binary Software: End User Software Moves to the Web}} die These, dass konventionelle Software zukünftig in immer größerem Maße durch webbasierte Applikationen ersetzt werde. Googles Linux-basiertes Betriebssystem \emph{Chrome OS} ist dafür ein gutes Beispiel. Nahezu alle Anwendungsprogramme werden hier im Webbrowser Chrome ausgeführt.
Zahlreiche in den letzten Jahren aufgekommene Standards wie \emph{Web Storage}, \emph{Web Sockets} und \emph{Media Capture} unterstreichen diesen Bedeutungszuwachs von Web-Anwendungen und stützen die Prognose Taivalsaaris. Ein Bereich, der sich bisher noch kaum im Web etablieren konnte und nach wie vor in erster Linie konventioneller Client-Software vorbehalten ist, ist die hardwarebeschleunigte Darstellung von 3D-Grafik.

In seinem Artikel \enquote{\emph{Is 3D Finally Ready for the Web?}} aus dem Jahr 2010 behandelt Sixto Ortiz Jr. die Fragestellung, ob das Web nach zahlreichen gescheiterten Anläufen inzwischen gerüstet sei, auch diese \enquote{letzte Bastion} \autocite{Taivalsaari:2011:Death_of_Binary_Software} traditioneller Software abzubilden. Der Autor analysiert hierfür den damaligen Stand webbasierter 3D-Grafik und kommt zum Schluß, dass es immer noch viele Hürden gibt, die einen Durchbruch dieser Technik verhindert. Er sah hierbei die Notwendigkeit von Browser-Plugins und den Mangel an Standardisierung als hauptsächliche Probleme.