\chapter{Motivation und Aufbau der Arbeit}
\label{CHAP:MOTIVATION}

Mit der Veröffentlichung des Internet Explorers 11 im Herbst 2013 \autocite{MS_RELEASE_IE11} ist der freie Grafikstandard \emph{Web Graphics Library} (WebGL) erst kürzlich in den Webbrowser von Microsoft eingezogen. Hierdurch sind die Grundvoraussetzungen für die Plugin-freie Darstellung hardwarebeschleunigter 3D-""Grafik nun in nahezu allen aktuellen Desktop-Browsern gegeben.
Auch im mobilen Bereich ist derzeit eine neue Dynamik zu erkennen. So kündigte Apple im Juni 2014 die Integration von WebGL in ihre mobile Plattform \emph{iOS 8} an \autocite{APPLE_WWDC_2014_WEBGL}. Der Einflussbereich dieser Technologie nimmt damit weiter zu und die Etablierung von 3D-""Grafik im \emph{World Wide Web} scheint in greifbare Nähe gerückt.

Vier Jahre nach Ortiz Artikel soll dessen Grundfrage, ob das WWW bereit sei für die dritte Dimension, erneut aufgegriffen werden. Gegenstand dieser Arbeit ist davon ausgehend die Evaluation der zwei offenen Grafikstandards  WebGL und X3D bezüglich ihrer Eignung für die Realisierung interaktiver 3D-""Grafik im Webbrowser. Hierfür werden die zwei Technologien hinsichtlich verschiedener Kriterien gegenüber gestellt.

Die Arbeit ist wie folgt strukturiert: Zunächst werden einige elementare Grundbegriffe und Prinzipien der Computergrafik erläutert, um das Verständnis der nachfolgenden Ausführungen zu erleichtern. Daraufhin wird die behandelte Anwendungsdomäne der betrachteten Klasse von 3D-""Anwendungen konkretisiert und der thematische Umfang der Untersuchung eingegrenzt. Ausgehend von exemplarischen Anwendungsfällen werden anschließend Ziele der Evaluation und Anforderungen einer solchen 3D-""Anwendung spezifiziert. Um den grundsätzlichen Aufbau und die Funktionsweise der zwei betrachteten Technologien aufzuzeigen, werden diese im Anschluss daran im Detail beleuchtet. Dasselbe Beispiel verdeutlicht jeweils die paradigmatischen Besonderheiten beider Ansätze. Schließlich wird die Evaluation ausgehend von den zuvor spezifizierten Kritieren durchgeführt. Hierfür werden zum einen die Ergebnisse automatisierter Tests betrachtet und zum anderen solche Aspekte untersucht, die nur schwer quantifiziert werden können. Zuletzt werden die Ergebnisse hinsichtlich der Zielvorgabe bewertet und ein Fazit gezogen.
